\subsection{MOtivation, data and research questions}

This study is motivated by the need to distinguish noise and signal in flame detectors,  create a model to eliminate noise being interpreted as signal by flame detector sensor channels.

 Describe the (a) motivation for your study, (b) the data and its suitability, and (c) the research questions you seek to investigate. Does the data need to be transformed? Your research questions should not simply be about describing data but should be linked to implications for a chosen application or domain. <400 words (<600 words if pairwork, so you can explain why there needs to be two of you.).
You'll be marked on the quality of the motivation, suitability of the data and the degree to which the research questions more than a mere description.

\subsection{Motivation}

introduction < 400 words

motivation

data

research questions: What features best characterise noise and signal and
Given this set of features of signal and noise charateristics, with is the optimal window size to apply smoothing algorithms, and which features contribute most to the model.

* the data

research questions

how to filter noise

The data is generated from flame detectors and presented in the form of log files, obtained during certification testing procedures. A log files being comprised of observations acquired every 10ms and stored in six bytes. The first three consisting of sensor channels Flame A, Flame B and Guard,

noise is being interpreted as signal, in the domain being investigated this mean that noise will be interpreted as the existence of fire within range of the detector, a scenario which must be avoided, the best case being it never occurring. 



\subsection{Domain overview}
Flame detectors are widely used in areas subject to fire hazards such as oil and gas installations and chemical plants. Detectors consist of hardware and software subject to SIL (Security Integrity Level) certification, aiming and reducing risk effects. Standards such as the IEC 61508\cite{wiki:IEC61508} are used to certify equipment and ensure compliance.  

\subsection{Problems to be tackled}

To carry out the analysis the barriers to overcome consist of:

\begin{itemize}
\item processing log files
\item choosing appropriate libraries
\item engineer features to inform our decisions
\item processing log files
\item choosing appropriate libraries
\item engineer features to inform our decisions

\end{itemize}

\subsection{Analytical questions}

The analytical questions we want to ask is how can signal be differentiated from noise? Though distance functions? Linear regression? A combination of both?

\subsection{Objectives}

\begin{itemize}
\item Trial different techniques
\item Find a suitable model to eliminate noise (smoothing algorithm)

Note, this could be a combination of existing algorithms, to make best use of the attributes present in available dataset.
\end{itemize}

\subsection{Data source} 

The data being analysed consists of data logs generated by commercially available flame detectors. Twenty eight files were examined in total, generated under test conditions. One log file (Test45.log) containing real fire data, some of the remaining logs containing RF (radio frequency) data erroneously reported in software as fire - this will be shown in attributes.

The logs files were obtained from a flame detector undergoing Functional Safety of Electrical/Electronic/Programmable Electronic Safety-related Systems, as defined in the IEC 61508 standard \cite{wiki:IEC61508}. One log file (Test45.log) containing real fire data, the other logs, some containing RF (radio frequency) data erroneously reported in software as fire.

\subsection{Analysis strategy}

Once our signal and noise have been characterized, our analysis strategy consists of engineering features, as well as creating models to quantify the levels of noise and signal. Distance functions, fft transforms and filters, such as proposed by Savitzky and Golay \cite{Savitzky:1964} provide a scheme to smooth signals, eliminating noise. In this work, we shall examine such schemes and compare the end results, determining by sch observations a filtering strategy to eliminate noise generated by GHz frequency interference.

WE FOUND THIS TABLE

\begin{table}[]
\begin{tabular}{|l|l|l|l|l|l|l|l|l|l|}
\hline
\multicolumn{10}{|l|}{Original data, filtered and interpolated} \\ \hline
Log file  &  CC & DR &  AR  & SGCC & SGDR & SGAR &  ICC & IDR & IAR \\ \hline
Test45.log & 0.9956 & 0.0234 & 0.0706 & 0.9979 & 0.9851 & 0.4314 & 0.9961 & 0.0000 & 0.0802 \\ \hline
Test46.log & 0.6518 & 0.0559 & 0.0759 & 0.7940 & 0.9876 & 0.4301 & 0.4763 & 0.0000 & 0.0852 \\ \hline
Test47.log & 0.8840 & 0.0815 & 0.0767 & 0.9254 & 0.9884 & 0.4116 & 0.5557 & 0.0000 & 0.0801 \\ \hline
Test48.log & 0.6908 & 0.2915 & 0.0763 & 0.6951 & 0.9938 & 0.3626 & 0.9439 & 0.0000 & 0.0603 \\ \hline
Test49.log & 0.1931 & 0.0278 & 0.0731 & 0.2154 & 0.9860 & 0.4462 & 0.2427 & 0.0000 & 0.0865 \\ \hline
Test50.log & 0.7170 & 0.1372 & 0.0770 & 0.7751 & 0.9925 & 0.4084 & 0.6453 & 0.0000 & 0.0808 \\ \hline
Test51.log & 0.8739 & 0.1808 & 0.0687 & 0.8827 & 0.9750 & 0.3541 & 0.9927 & 0.0000 & 0.0648 \\ \hline
Test52.log & 0.3417 & 0.3926 & 0.0889 & 0.3394 & 0.9940 & 0.3836 & 0.9083 & 0.0000 & 0.0546 \\ \hline
Test53.log & 0.9146 & 0.1998 & 0.0705 & 0.9220 & 0.9801 & 0.3679 & 0.9914 & 0.0000 & 0.0614 \\ \hline
Test54.log & 0.9172 & 0.2256 & 0.0661 & 0.9275 & 0.9796 & 0.3624 & 0.9771 & 0.0000 & 0.0582 \\ \hline
Test55.log & 0.8123 & 0.0951 & 0.0747 & 0.8739 & 0.9891 & 0.4100 & 0.5765 & 0.0000 & 0.0802 \\ \hline
Test56.log & 0.8891 & 0.1309 & 0.0738 & 0.9020 & 0.9895 & 0.3848 & 0.8978 & 0.0000 & 0.0706 \\ \hline
Test57.log & 0.3361 & 0.2406 & 0.0801 & 0.3370 & 0.9923 & 0.4026 & 0.7804 & 0.0000 & 0.0728 \\ \hline
Test58.log & 0.2004 & 0.0276 & 0.0753 & 0.2245 & 0.9869 & 0.4415 & 0.2729 & 0.0000 & 0.0880 \\ \hline
Test59.log & 0.3922 & 0.0418 & 0.0720 & 0.5423 & 0.9888 & 0.4339 & 0.3553 & 0.0000 & 0.0825 \\ \hline
Test60.log & 0.4395 & 0.0635 & 0.0755 & 0.5177 & 0.9888 & 0.4231 & 0.4593 & 0.0000 & 0.0878 \\ \hline
Test61.log & 0.1974 & 0.0208 & 0.0752 & 0.1916 & 0.9876 & 0.4432 & 0.2653 & 0.0000 & 0.0877 \\ \hline
Test62.log & 0.4155 & 0.0284 & 0.0720 & 0.5919 & 0.9872 & 0.4395 & 0.4111 & 0.0000 & 0.0857 \\ \hline
Test63.log & 0.5986 & 0.0428 & 0.0750 & 0.7438 & 0.9883 & 0.4231 & 0.4770 & 0.0000 & 0.0872 \\ \hline
Test64.log & 0.5219 & 0.0382 & 0.0733 & 0.7022 & 0.9877 & 0.4300 & 0.3897 & 0.0000 & 0.0847 \\ \hline
Test65.log & 0.2200 & 0.0265 & 0.0722 & 0.2588 & 0.9866 & 0.4427 & 0.2898 & 0.0000 & 0.0863 \\ \hline
Test66.log & 0.3489 & 0.0536 & 0.0762 & 0.4134 & 0.9884 & 0.4316 & 0.3954 & 0.0000 & 0.0901 \\ \hline
Test67.log & 0.8746 & 0.1965 & 0.0750 & 0.8895 & 0.9916 & 0.3934 & 0.7683 & 0.0000 & 0.0717 \\ \hline
Test68.log & 0.7765 & 0.1570 & 0.0758 & 0.7922 & 0.9910 & 0.3969 & 0.7790 & 0.0000 & 0.0743 \\ \hline
Test69.log & 0.2566 & 0.3190 & 0.0866 & 0.2577 & 0.9931 & 0.3887 & 0.9030 & 0.0000 & 0.0569 \\ \hline
Test70.log & 0.4387 & 0.0304 & 0.0738 & 0.6178 & 0.9884 & 0.4256 & 0.4610 & 0.0000 & 0.0863 \\ \hline
Test71.log & 0.6141 & 0.0445 & 0.0733 & 0.7820 & 0.9865 & 0.4282 & 0.4476 & 0.0000 & 0.0848 \\ \hline
Test72.log & 0.7652 & 0.2592 & 0.0772 & 0.7695 & 0.9930 & 0.3627 & 0.9456 & 0.0000 & 0.0672 \\ \hline

\end{tabular}
\caption{My table}
\label{table:kysymys}
\end{table}

Table \ref{table:kysymys} on page \pageref{table:kysymys} refers to the ...