\section{Tasks and approach}

Of special importance to this study is the fact that of the 28 log files, \textit{Test45.log} was generated by a UUT exposed to a flame while subject to no RF interference, and this file should be our reference to characterise signal, while all remaining files were generated by UUTs exposed to RF interference and no flame, and should be our references to charaterise noise.

With that in mind, our tasks are:

\subsection{Feature engineering}

To proceed with the remaining tasks, a number of features need to be created first, based on Channel A and Channel B values for every log these involve

\begin{itemize}
    \item Log file name
    \item Channel 1 (flame a) raw data
    \item Channel 2 (flame b) raw data
    \item Channel 1 normalised data
    \item Channel 2 normalised data
    \item Normalised distance function
    \item Antiphase function
    \item Normalised correlation coefficient
    \item Filtered (Savitzy-Golay filter) channel 1 normalised data
    \item Filtered (Savitzy-Golay filter) channel 2 normalised data
    \item Filtered distance function
    \item Filtered antiphase function
    \item Filtered correlation coefficient
    \item Interpolated channel 1 normalised data
    \item Interpolated channel 2 normalised data
    \item Interpolated distance function
    \item Interpolated antiphase function
    \item Interpolated correlation coefficient
\end{itemize}

\subsection{Characterising signal and noise}

To characterise noise and signal we shall plot Flame A and Flame B values for our flame reference file, compared to another file then compare plots.

\subsection{Antiphase function}

The computational method for our antiphase function can be expressed by:

$$  A \Rightarrow \frac{\Delta Fa}{\Delta Fb}<0$$
where
$$ \Delta Fa = Fa{_i}-Fa{_{i-1}}, \Delta Fb = Fb{_i}-Fb{_{i-1}}, \forall i > 1 $$

The sign of delta ratios indicating phase (positive) or antiphase (negative).

Our visual method is to plot detected antiphase together the time series for the flame channels.

\subsection{Distance function}

We use a distance function defined as:

$$ Fd = |Nfa - Nfb| $$

Where Fd is the absolute difference between channels a and b. Visually distance between channels is displayed as a plot. 

\subsection{Correlation coefficient}

We extract a correlation coefficient \textbf{between} channel 1 (x) and channel 2 (y) - it is important to noticed here that we are using the correlation coefficient to determine how close the signals are to each other and not how one might be affecting the other.

$$ r =\frac{\sum ^n _{i=1}(x_i - \bar{x})(y_i - \bar{y})}{\sqrt{\sum ^n _{i=1}(x_i - \bar{x})^2} \sqrt{\sum ^n _{i=1}(y_i - \bar{y})^2}} $$

We will use the output to have all coefficients in tabular form for comparison, as well as headers for sample plots.

\subsection{Savistzky-Golay filter}

We use a Savitzky-Golay filter \cite{Savitzky:1964} defined as:

$$ Y_j= \sum _{i=\tfrac{1-m}2}^{\tfrac{m-1}2}C_i\, y_{j+i},\qquad  \frac{m-1}{2} \le j \le n-\frac{m-1}{2} $$

where x is an independent variable and yj is an observed value. Note we are applying the filter \textbf{independently} to each of the channels a and b signals, to then obtain a correlation coefficients as well as additional engineered features.

This will used to generate data for linear regression plots as well as correlation coefficients.

\subsection{Linear Interpolation}

We use a linear interpolation algorithm, for all observations where the distance between channels, at the observation index, is greater than a threshold \textit{t}, defined as:

$$ y = \frac{y_0(x_1-x)+y_1(x-x_0)}{x_1-x_0} , \forall Fd > t $$

Notice this is also applied, as the previous filter, on a per-channel basis, x being the independent and y being the dependent variable.
This will used to generate data for linear regression plots as well as correlation coefficients. 

NB all normalisation is done using minimum and maximum values scaled to byte values (0-255):

$$ z{_i}=\frac{x{_i}-min}{max-min} , max-min \neq 0 $$