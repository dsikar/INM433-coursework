%% INTRODUCTION

This study if motivated by the need to investigate strategies to process signals subject to RF (radio frequency) interference, eliminating what may be interpreted as noise, while preserving the signal. We investigating in particular commercially available Flame Detectors, taking into consideration approval processes related to IEC61508 \cite{wiki:IEC61508} SIL (Safety Integrity Level) approvals, which provide certification, as required by the insurance sector, to ensure compliance.

The data analysed in this study consists of 28 log files generated by UUTs (units under test) in a certified testing facility, where the UUT is place in a chamber where it is subjected to RF interference. Tests last up till five minutes, with acquisitions made at 10ms intervals. Each acquisition comprising of the analogue-to-digital converted signals of 3 sensors; 2 being redundant flame detectors and 1 being a "guard" which detects \textit{blackbody radiation} \cite{Massoud:2005} and can be used to avoid false positives i.e. heat being incorrectly identified as a flame.

Each observation (log entry) consists of 6 bytes, one for the Flame A channel, one for the Flame B channel and one for the Guard Channel. The value being an unsigned byte, ranging from 0 to 255. The fourth byte consisting of two 4 unsigned bit values ranging from 0 to 15. These are the voltage references for the flame sensors. Byte 5 is not used and byte 6 consists of three flags (bit values), one unused bit value and a four bit rolling counter to ensure readings are continuous. 

The data is retrieved from the UUT internal storage as a plain text file, which each acquisition stored in hexadecimal format, which must be processed by converting each byte into a base 10 number, with further masking to retrieve flags and four bit values.

Once the data is presented in a familiar decimal format, the research questions we would like to ask is how do different signal processing strategies perform in attenuating noise? What metrics can be used to quantify noise and can the same metrics be used to determine the suitability of one strategy over another?
